\begin{abstract}

    Производственная практика была проведена в депертаменте математики,
    механики и компьютерных наук.
    Темой исследования стали позиционные диференциальные игры.
    Оптимальной стратегия в подобных задачах
    строится методом экстремального сдвига на множество,
    называемое максимальным стабильным мостом.
    Таким образом, решение игровой задачи сводится к построению такого множества.

    Цель данной работы - получить численные методы для аппроксимации стабильного моста.
    Был рассмотрен подклас линейных игр без фазовой переменной в правой части.

    Метод стабильных мостов, будучи потенциально весьма конструктивным,
    доныне не обрел значимого практического применения.
    Реальные механические системы имеют большую размерность,
    поэтому даже аппроксимации стабильных мостов трудновычислимы.
    Для решения подобных задач на практике используются
    субоптимальные(инженерные, эвристические) методы.
    Тем не менее, данный метод может быть приминим для решения некоторых простых задач.

    С учетом постоянного роста вычислительных мощностей, 
    численный подход к решению расматриваемых задач
    способен увеличивать свою актуалность.

    Результатом иследовательской работы,
    стала программа написанная на языке С++,
    инкапсулируящая в себе работу предложенного алгоритма. 

\end{abstract}