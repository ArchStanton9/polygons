\sloppy

% Настройки стиля ГОСТ 7-32
\EqInChapter % формулы будут нумероваться в пределах раздела
\TableInChapter % таблицы будут нумероваться в пределах раздела
\PicInChapter % рисунки будут нумероваться в пределах раздела

\author{Кощеев Никита}
\title{Отчет о производственной практике}
\date{\today}

\usepackage{pscyr}
\renewcommand{\rmdefault}{ftm}

\usepackage[
bookmarks=true, colorlinks=true, unicode=true,
urlcolor=black,linkcolor=black, anchorcolor=black,
citecolor=black, menucolor=black, filecolor=black,
]{hyperref}

\usepackage{graphicx}
\graphicspath{ {images/} }

\geometry{right=20mm}
\geometry{left=30mm}

% Произвольная нумерация списков.
\usepackage{enumerate}

% ячейки в несколько строчек
\usepackage{multirow}

% itemize внутри tabular
\usepackage{paralist,array}

%\usepackage{enumerate}
%\setcounter{tocdepth}{2}

\newcommand{\dimension}{\mathbb{R}^n}