\chapter{Постановка задачи}

\section{Математическая модель}

Рассмотрим задачу сближения-уклонения двух материальных точек
в пространстве $\dimension$: 

\begin{equation}
    \label{eq:system_main}
    \frac{dx}{dt} = f_1(x,t,u) + f_2(x,t,v)
\end{equation}
 
\begin{itemize}
    \item $t \in [0,\theta]$, где $\theta$ - момент окончания игры
    \item $x \in \dimension$ - фазовый вектор
    \item $u(t) \in P \subset \dimension$ - управление первого игрока
    \item $v(t) \in Q \subset \dimension$ - управление второго игрока
\end{itemize} 

Выбор правой части $F(x,t,u,v) = f_1(x,t,u) + f_2(x,t,v)$
в уравнении \ref{eq:system_main} обеспечит нам выполнение условия Айзекса:

\begin{equation}
    \label{eq:isaacs}
    \forall s, x \in \dimension :
    \min \limits_{u \in P} \max \limits_{v \in Q}
    (s, F(x,t,u,v)) =
    \max \limits_{v \in Q} \min \limits_{u \in P}
    (s, F(x,t,u,v)) 
\end{equation}
 
В таком случае в маленькой игре будет существовать седловая точка,
по этому достаточно рассматривать дифференциальную игру
в классе позиционных стратегий.
В этом случае стратегии первого игрока зависят
только от реализовавшейся позиции.

При помощи фундаментальной матрицы Коши,
мы можем сделать замену переменных
и перейти к эквивалентной дифференциальной игре
без фазовой переменной в правой части.
В итоге получим систему:

\begin{equation}
    \label{eq:system_plain}
    \frac{dy}{dt} = B(t)u + C(t)v
\end{equation}

Для простоты будем считать 𝐵(𝑡) и 𝐶(𝑡) матрицами
размерности соответствующей нашей системе, т.е. $2\times2$

\section{Максимальный стабильный мост}

При выполнении условия \ref{eq:isaacs}
гарантировано существуют оптимальные стратегии
$u^0$ и $v^0$: 

\begin{equation}
    \exists u^0(t,x,\epsilon), v^0(t,x,\epsilon):
    [t_0, \theta] \times \dimension \times \mathbb{R}^+
    \rightarrow \dimension
\end{equation}

По теореме об алтернативе существует
максимальный стабильный мост W:

\begin{equation}
    W \subset [t_0,\theta] \times \dimension,
\end{equation}

\begin{itemize}
    \item применение первым игроком управления $u^0(t,x,\epsilon)$
    при $x_0 \in W(t_0)$ гарантирует $x(\theta) \in M + O(\epsilon)$
    \item применение первым игроком управления $v^0(t,x,\epsilon)$
    при $x_0 \notin W(t_0)$ гарантирует $x(\theta) \notin M \ominus O(\epsilon)$
\end{itemize}

Для получения апроксимации стабильного моста
для системы \ref{eq:system_plain} возьмем разбиение:
\begin{equation}
    \{t_i\}, t_n = \theta
\end{equation}

Управления игроков порождаются как кусочно-постоянные функции,
терпящие разрывы в $𝑡_i$.
Каждое последуеще сечение моста вычисляется по следующей формуле:

\begin{eqnarray}
    \begin{aligned}
        &W_n = M\\
        &W_{i-1} =
        \left(W_i+\int_{t_i}^{t_{i-1}} B(t) dt \cdot P\right)
            \ominus
        \int_{t_i}^{t_{i-1}} C(t) dt \cdot Q
    \end{aligned}
\end{eqnarray}
